\chapter{Описание стендового оборудования и сбор данных}

\section{Состав стенда}
Структурная схема стенда для исследования асинхронного электропривода приведена на рисунке \ref{fig:test_bench_scheme}.


\begin{figure}[h!]
    \centering
    \includegraphics[width=0.8\textwidth]{../pics/test-bench.png}
    \caption{Структурная схема стендового оборудования для исследования асинхронного электропривода}
    \label{fig:test_bench_scheme}
\end{figure}

В составе стенда входит следующее основное оборудование:
\begin{itemize}
\item Асинхронный двигатель ЕЛПРОМ ТРОЯН 0,75кВт
\item Тормоз FZ25K-3
\item Подшипниковый узел
\item Упругая муфта
\item Частотный преобразователь Веспер ES-8200SIL
\item Датчики тока ACS712
\end{itemize}


Объектом исследования является асинхронный двигатель. Он подключен к частотному преобразователю, который обеспечивает питание двигателя переменным током с регулируемой частотой. Тормоз используется для создания нагрузки на вал двигателя, что позволяет моделировать различные режимы работы. Упругая муфта и подшипниковый узел соединяют вал двигателя с тормозом, позволяя передавать крутящий момент при различных условиях нагрузки. Датчики тока ACS712 установлены на фазах статора двигателя для измерения токов в реальном времени.

\section{Параметры измерений и методика сбора данных}
Для проведения исследований был проведен сбор экспериментальных данных сигналов фазного тока асинхронного двигателя при различных технических состояниях. Регистрация двух фаз тока статора (каналы A и B) осуществлялась с частотой дискретизации 10 кГц, при этом длительность каждой записи составляла 10 с, что соответствует 100 000 отсчетов на каждый зарегистрированный сигнал. Испытания проводились при электрической частоте питания 20 Гц в двух режимах нагрузки: холостой ход и нагрузка, соответствующая 50\% от номинальной мощности двигателя.


Было исследовано четыре диагностируемых состояния электродвигателя. \textbf{Нормальная работа} соответствовала исправному двигателю, отключенному от муфты стенда. Состояние \textbf{дисбаланса ротора} достигалось подключением двигателя к муфте, масса которой создавала неуравновешенность. \textbf{Перекос ротора (несоосность)} имитировался установкой дополнительной шайбы толщиной 0,3 мм под опорным подшипниковым узлом. \textbf{Износ подшипников} моделировался путем установки подшипника с искусственным дефектом на внешнем кольце в двигатель.


Для обеспечения статистической достоверности каждый эксперимент повторялся многократно (табл. \ref{tab:dataset_summary}). Испытания проводились по фиксированному протоколу: разгон двигателя до установившегося режима в течение 10-15 секунд, запись сигнала длительностью 10 с и сохранение метаданных эксперимента (класс дефекта, режим работы, дата/время).

Для увеличения объема выборки и приведения данных к единому размеру, необходимому для последующего анализа, каждый зарегистрированный сигнал подвергался процедуре сегментации. Исходные временные ряды разбивались на последовательные отрезки с использованием скользящего окна длиной 4096 отсчетов и перекрытием 50\% между соседними сегментами. Данная процедура позволила сформировать массив стандартизированных фрагментов сигналов для дальнейшей обработки.

\section{Методика извлечения признаков во временной области}
Для каждого сегмента сигнала фазного тока, полученного в процессе сегментации, вычислялся набор статистических признаков, отражающих форму, энергию и вариативность сигнала во временной области. Эти характеристики позволяют количественно описывать основные свойства сигнала без перехода в частотную область и широко применяются в задачах диагностики технического состояния электрических машин. В работе были использованы следующие признаки: среднеквадратичное значение, коэффициент асимметрии, коэффициент эксцесса, размах, коэффициент импульсности и коэффициент формы [].

\section{Методика извлечения признаков в частотной области}
Для анализа частотных характеристик сигналов фазных токов использовались методы спектральной и огибающей обработки. Предварительный анализ амплитудно-частотных спектров показал наличие выраженных гармонических составляющих, обусловленных широтно-импульсной модуляцией (ШИМ) инвертора. На спектрограммах (рис. \ref{fig:spectrogram_pwm}) отчётливо наблюдались две доминирующие гармоники несущей на частотах около \textbf{1665 Гц} и \textbf{3330 Гц}, не связанных с механическим состоянием двигателя. Их присутствие маскировало диагностически значимые низкочастотные составляющие, что делало прямое применение преобразования Фурье неинформативным.

В связи с этим для выделения информативной низкочастотной компоненты огибающей тока была применена методика демодуляции с использованием преобразования Гильберта. Обработка каждого сегмента сигнала включала следующие этапы:

Обработка каждого сегмента сигнала включала следующие этапы:

\begin{enumerate}
    \item \textbf{Полосовая фильтрация.}  
    Сигнал \( x(t) \) пропускался через полосовой фильтр Баттерворта четвёртого порядка с центральной частотой \( f_c = 1670~\text{Гц} \) и полосой пропускания \( \pm 50~\text{Гц} \):
    \[
        x_{\mathrm{f}}(t) = \mathrm{BPF}\{x(t)\}, \quad 1620~\text{Гц} \le f \le 1720~\text{Гц}.
    \]

    \item \textbf{Преобразование Гильберта.}  
    Для вычисления мгновенной амплитуды (огибающей) формировался аналитический сигнал:
    \[
        z(t) = x_{\mathrm{f}}(t) + j\,\mathcal{H}\{x_{\mathrm{f}}(t)\},
    \]
    где \( \mathcal{H}\{\cdot\} \)~— преобразование Гильберта.  
    Огибающая сигнала определяется как:
    \[
        e(t) = |z(t)| = \sqrt{x_{\mathrm{f}}^2(t) + \mathcal{H}^2\{x_{\mathrm{f}}(t)\}}.
    \]

    \item \textbf{Фильтрация низких частот.}  
    Для устранения остаточных высокочастотных колебаний применялся фильтр нижних частот (ФНЧ) с граничной частотой \( f_{\text{LP}} = 200~\text{Гц} \), обеспечивающий выделение диагностически значимых низкочастотных составляющих огибающей.

    \item \textbf{Спектральный анализ.}  
    Для оценки спектральной плотности мощности огибающего сигнала использовался метод Вэлча с окном Ханна длиной 512 отсчётов и перекрытием 50\% (256 отсчётов).  
    Итоговая оценка спектральной плотности мощности вычислялась по формуле:
    \[
        P_{xx}(f) = \frac{1}{K}\sum_{k=1}^{K} \frac{|\mathrm{FFT}\{w(n)x_k(n)\}|^2}{U},
    \]
    где \( w(n) \)~— окно Ханна, \( U \)~— нормировочный коэффициент энергии окна, \( K \)~— число сегментов.

    \item \textbf{Извлечение признаков.}  
    На основе спектра огибающей вычислялся набор диагностических признаков: \textbf{(Взять из наиболее значимых!)}
\end{enumerate}
