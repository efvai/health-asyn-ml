\chapter*{Примерное содержание}

\begin{enumerate}
    \item Рассматриваются этапы подготовки данных, выбор и обучение модели, а также оценка её качества.
    \item В качестве данных используются временные ряды, полученные с датчиков тока асинхронного электродвигателя.
    \begin{enumerate}
        \item Приводится описание и структурная схема стендового оборудования, предназначенного для сбора данных.
        \item Описываются режимы работы электропривода, виды имитируемых неисправностей, а также процесс сбора данных (частота дискретизации, длительность записи и т.д.).
        \item Описываются методы предварительной обработки данных, включая фильтрацию шумов.
    \end{enumerate}
    \item Приводится методика извлечения признаков из временных рядов, включая статистические характеристики, и частотные признаки.
    \begin{enumerate}
        \item Описываются алгоритмы выделения признаков, такие как преобразование Фурье (но с использованием welch), разбиение временных признаков на окна, разбиение частотной области на полосы, приводится спектрограмма.
        \item Приводятся некоторые формулы для расчета признаков.
    \end{enumerate}
    \item Рассматриваются процесс обучения модели случайного леса, включая выбор гиперпараметров и методы кросс-валидации. Рассматривается вычисления важности признаков методами SHAP, MDI и Permutation Importance.
    \begin{enumerate}
        \item Приводится описание алгоритма случайного леса, его преимущества и недостатки.
        \item Приводятся результаты обучения модели, включая метрики качества (точность, полнота, F1-мера и т.д.).
        \item Приводятся графики важности признаков, производится переобучение модели на отобранных признаках.
        \item Обсуждаются результаты, включая интерпретацию важности признаков, возможный вклад и физический смысл наиболее значимых признаков. Сравниваются результаты методов оценки важности признаков.
    \end{enumerate}
    \item Приводится заключение, включая основные выводы и направления для будущих исследований.
\end{enumerate}

\chapter*{Введение}

Асинхронные двигатели находят широкое применение в различных отраслях промышленности, таких как металлургия, нефтехимия, водоснабжение, транспорт и производство. Например, они используются для привода насосов, вентиляторов, конвейеров, компрессоров и других механизмов, требующих надежного и эффективного электропривода. Несмотря на высокую надежность, асинхронные двигатели подвержены различным неисправностям. Наиболее распространённой причиной отказов являются повреждения подшипников, на долю которых приходится до половины всех случаев отказа электродвигателей [1]. Кроме того, встречаются электрические неисправности (межвитковые замыкания, пробой изоляции, обрыв фазы и т.д), а также механические дефекты (дисбаланс, перекос ротора и т.д) и нарушения в питающей сети. Эти неисправности могут привести к простою производства, снижению качества продукции и значительным финансовым потерям. В связи с этим особое значение приобретает применение современных и эффективных методов мониторинга состояния, позволяющих своевременно выявлять и диагностировать неисправности асинхронных электродвигателей.

В настоящее время реактивное обслуживание постепенно заменяется мониторингом состояния и предсказательным обслуживанием, что позволяет автоматизировать оценку состояния электродвигателя и снизить зависимость от трудоёмких ручных проверок. Совершенствование методов мониторинга и диагностики способствует раннему выявлению отклонений и предотвращению простоев оборудования, что, в свою очередь, уменьшает затраты на обслуживание и повышает надёжность системы. По этой причине мониторинг состояния электроприводов получил значительное внимание в научной литературе [2--5].

Существуют два основных подхода к диагностике состояния электроприводов: модельно-ориентированный и основанный на данных. Модельно-ориентированный подход использует математические модели, описывающие физические процессы и динамику системы. Такие методы могут обеспечивать высокую точность, однако требуют глубоких знаний о конструкции, параметрах и принципах работы оборудования, что ограничивает их применение в реальных промышленных условиях. С другой стороны, методы, основанные на данных, включают современные технологии искусственного интеллекта, такие как искусственные нейронные сети (ИНС), классические алгоритмы машинного обучения и методы нечеткой логики. Эти подходы приобрели широкую популярность в последние годы и особенно эффективны, когда имеется доступ к данным с датчиков, но математическая модель системы неизвестна или слишком сложна для построения. Алгоритмы машинного обучения позволяют решать задачи классификации и прогнозирования состояния оборудования, обеспечивая высокую гибкость и адаптивность к различным типам неисправностей.

В задачах мониторинга и диагностики состояния электроприводов наибольшее распространение получили анализ вибрационных сигналов и фазных токов. Вибрационные данные традиционно применяются для выявления механических неисправностей, в то время как анализ фазных токов эффективен как для обнаружения электрических дефектов, так и для косвенной оценки механических, поскольку они отражаются на электромагнитных процессах. Для обработки таких сигналов широко используются методы анализа во временной, частотной и временно-частотной областях (например, преобразование Фурье, вейвлет-преобразование и преобразование Хильберта–Хуанга). Подходы MCSA (Motor Current Signature Analysis) и вибрационного анализа (Vibration Signature Analysis, VSA) основаны на выделении информативных спектральных признаков, характеризующих состояние привода. Помимо вибрации и токов, дополнительными источниками информации могут служить температура, скорость вращения, акустические сигналы и параметры питающей сети.

Для снижения влияния шума и выделения информативных признаков, отражающих рабочие и неисправные состояния электропривода, выполняется предварительная обработка сигналов. На этом этапе могут применяться методы заполнения пропусков, фильтрации, нормализации и другие процедуры улучшения качества данных. Полученные признаки формируют основу для построения модели классификации и последующего анализа состояния оборудования.

Для повышения эффективности и интерпретируемости модели применяется отбор признаков (feature selection). Он позволяет исключить избыточные и коррелированные параметры, сократить размерность пространства признаков и тем самым повысить устойчивость моделей к переобучению, улучшить её обобщающую способность и уменьшить вычислительные затраты. В качестве фильтрационных подходов могут использоваться порог дисперсии и анализ кросс-корреляции между признаками; для снижения размерности — метод главных компонент (Principal Component Analysis, PCA). Оценка значимости признаков осуществляется с помощью современных методов интерпретации, таких как MDI (Mean Decrease in Impurity), Permutation Importance и SHAP values.

После формирования оптимального набора признаков осуществляется построение модели классификации, предназначенной для распознавания состояния электропривода. В задачах технической диагностики применяются как классические алгоритмы машинного обучения, так и современные методы глубокого обучения. К наиболее распространённым относятся Random Forest, Support Vector Machine (SVM), k-Nearest Neighbors (kNN), Gradient Boosting и различные модификации нейронных сетей. Выбор конкретной модели определяется особенностями исходных данных, объёмом обучающей выборки и требованиями к интерпретируемости и вычислительным ресурсам. \colorbox{yellow}{[ОБЗОР СТАТЕЙ]} Для оценки качества классификаторов обычно применяются стандартные метрики — точность (accuracy), полнота (recall), точность положительных предсказаний (precision) и интегральный показатель F1-score.

Цель данной работы — оценить возможности диагностики механических неисправностей асинхронного электродвигателя путем анализа фазных токов. Экспериментальные данные, представляющие собой временные ряды фазных токов в разных режимах работы и при различных неисправностях, были получены на экспериментальном стенде. В работе описан процесс их обработки, а также формирования информативных признаков в двух областях: временной и частотной. На основе извлечённых признаков была обучена модель классического машинного обучения Random Forest (RF), и проведена оценка точности её классификации по основным метрикам. Центральное место в исследовании занял сравнительный анализ важности признаков, проведенный тремя методами (SHAP, MDI, Permutation Importance). Это позволило не только подтвердить эффективность подхода, но и выявить набор наиболее значимых признаков, напрямую характеризующих механическое состояние электропривода.





